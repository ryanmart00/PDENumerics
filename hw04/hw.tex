\documentclass[11pt]{article}
\usepackage[tmargin=1.25in,lmargin=.25in,rmargin=1in,bmargin=1in,paper=letterpaper]{geometry}
\usepackage{amsmath,amssymb, amsfonts}
\usepackage{multirow, color}
\usepackage{mathrsfs} % for \mathscr{S}
\usepackage{fancyhdr,ifthen,lastpage}
\usepackage[utf8]{inputenc}
\usepackage{textgreek}
\usepackage{verbatim}
% Begin My Stuff {{{
\usepackage{ifthen}
\usepackage{amsfonts}

\usepackage{tikz}
\usetikzlibrary{calc,shapes}
\usetikzlibrary{decorations.markings,arrows,positioning}
\newcommand{\tikzmark}[1]{\tikz[overlay,remember picture] \node (#1) {};}
\usepackage{xparse}
\usepackage{cancel}

\usepackage{etoolbox}
\newcounter{tikzmarkcounter}

\let\bb\mathbb
\def\Re{\text{Re}}
\def\Im{\text{Im}}
\def\tr{\text{tr}}
\def\Id{\text{Id}}
\def\supp{\text{supp }}
\def\sgn{\text{sgn}}
\def\O{\mathcal O}
\let\p\partial

\newcommand{\prob}{ \hfill \tikzmark{br\thetikzmarkcounter}
    \tikz[overlay,remember picture]{\draw[black]
    ($(bl\thetikzmarkcounter)+(-0.2em,1.6em)$) rectangle
    ($(br\thetikzmarkcounter)+(1.0em,-0.9em)$);}
    \refstepcounter{tikzmarkcounter}~\newline}

\makeatletter
\let\latexitem\item
\NewDocumentCommand{\problemitem}{o}{%
  \IfValueT{#1}{\latexitem[\tikzmark{bl\thetikzmarkcounter}\formatproblem{#1}] }%
  \IfNoValueT{#1}{\ifthenelse{\@enumdepth=1}{\stepcounter{enumi}
    \latexitem[\tikzmark{bl\thetikzmarkcounter}\formatproblem{\letter\arabic{enumi}}]}
    {\stepcounter{enumii} \latexitem[\tikzmark{bl\thetikzmarkcounter} (\alph{enumii})]}}
}
\makeatother
\NewDocumentCommand{\formatproblem}{m}{\textbf{#1:}}
\NewDocumentEnvironment{problems}{O{}}
 {\begin{enumerate}}
 {\end{enumerate}}

\newcommand{\diff}[2][]{\mathop{\frac{d #1}{d #2}} }
\newcommand{\pdiff}[2][]{\mathop{\frac{\partial #1}{\partial #2}} }
\newcommand{\conj}[1]{\mathop{\overline{#1}} }
\newcommand{\abs}[1]{{\mathop{\left\lvert #1 \right\rvert}} }
\newcommand{\norm}[1]{{\mathop{\left\lvert\left\lvert #1 \right\rvert\right\rvert}} }
\newcommand{\col}[1]{\left[ \begin{array}{c} #1 \end{array} \right]}
\newcommand{\bemph}[1]{\textbf{\textit{#1}} }
\usepackage{scalerel,stackengine}
\stackMath
\newcommand\what[1]{%
\savestack{\tmpbox}{\stretchto{%
  \scaleto{%
    \scalerel*[\widthof{\ensuremath{#1}}]{\kern.1pt\mathchar"0362\kern.1pt}%
    {\rule{0ex}{\textheight}}%WIDTH-LIMITED CIRCUMFLEX
  }{\textheight}%
}{2.4ex}}%
\stackon[-6.9pt]{#1}{\tmpbox}%
}


% End My Stuff
% }}}
\def\Div{\text{div}}
\let\div\Div

\def\Vol{\text{Vol}}

\begin{document}
\pagestyle{fancy}

%% HEADER %%
\lhead{ Math 228B - Spring \the\year}
\rhead{Ryan Martinez}
\chead{\bf Homework 4 }

%% FOOTER %%
\lfoot{} 
\rfoot{}
\cfoot{}

\begin{center}
 {\Large\bf Math 228B: Homework 4}
\end{center}

~

\begin{problems}
    \problemitem[1a] Derive the Galerkin formulation of 
    $$\partial^4_x u = 480x - 120$$
    $$u(0)= u'(0) = u(1) = u'(1) = 0$$
\prob

If there was a function $u$ defined on $[0,1]$ which satisfies the above PDE 
then it must be the case that for any function $\phi$ with 
$\phi(0) = \phi'(0) = \phi(1) = \phi'(1) = 0$ that 
$$\int_0^1 \partial^4_x u \phi dx = \int_0^1 f \phi dx$$
Integrating by parts twice we must also have 
$$\int_0^1 \partial^2_x u \partial^2_x \phi dx 
+ \partial^3 u \phi|_0^1 - \partial^2_x u \partial_x \phi|_0^1= \int_0^1 f \phi dx$$
but since we chose $\phi$ to be 0 to first order on the boundaries this is 
$$\int_0^1 \partial^2_x u \partial^2_x \phi dx = \int_0^1 f \phi dx.$$
Now, this will determine $u$ uniquely if we check this against enough $\phi$ so that 
$\partial^2 \phi$ span a dense subspace of the image of $\partial^2_x u$ on the space 
we look for a solution $u$, \emph{as long as $\partial^2_x$ is }

\end{problems}
\end{document}
